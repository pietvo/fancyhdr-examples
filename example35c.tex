\documentclass{article}

\usepackage{lipsum}
\usepackage{boxedminipage}
\usepackage{fancyhdr}
\usepackage{extramarks}
\fancyhead{}
\fancyfoot[C]{\thepage}
\fancypagestyle*{intro}{%
  \renewcommand{\headrule}{}
  \fancyhead[R]{INTRODUCTION}
}
\fancypagestyle{contents}[intro]{\fancyhead[R]{CONTENTS}}

\fancyhead[L]{\firstleftxmark}
\newcommand\rightheader{\firstrightxmark}
\fancyhead[R]{\rightheader}
\fancypagestyle*{fancy}{}

\extramarksnewmark{which}

\newcommand\checkrightreset{%
  \ifextramarksmissing{left}
  {}
  {% if there is a sectionmark on the page
    \ifnum\extramarkslast{which}=0
      % that is the last (no subsection following on the page)
      % then reset the subsection mark
      \extramarksreset{right}%
    \fi
  }%
}

\renewcommand{\sectionmark}[1]{%
  \extramarksleft{\thesection. #1}%
  \extramarksput{which}{0}%
  \renewcommand\rightheader{\checkrightreset\firstrightxmark}%
}

\renewcommand{\subsectionmark}[1]{%
  \extramarksright{\thesubsection\ #1}%
  \extramarksput{which}{1}%
}

\begin{document}
\pagestyle{intro}
\thispagestyle{contents}
\tableofcontents

\bigskip

\noindent
\begin{boxedminipage}{\textwidth}
This is example 35c, a variant of example 35 in the \texttt{fancyhdr} documentation.

{\em In this variant we have the final code of example 35 directly in the preamble, without the intermediate stages. This is easier for later reference. It also means that the `flaw' in section~\ref{sec:flaw}, where there was an ``ugly inheritance'', is no longer there. This is like example 35b, but this version is simplified to omit the check whether a subsection starts at the second half of the page.}

In this document we put the first section title (left) and the first subsection title (right) in the page headers. This is hard (probably impossible) to achieve with the standard \LaTeX{} marks.

\textbf{NOTE: This requires the new \texttt{extramarks.sty} (version 5.0 or later).}

\begin{verbatim}
\usepackage{fancyhdr}
\usepackage{extramarks}
\fancyhead{}
\fancyhead[L]{\firstleftxmark}
\newcommand\rightheader{\firstrightxmark}
\fancyhead[R]{\rightheader}
\fancyfoot[C]{\thepage}
\end{verbatim}
\end{boxedminipage}

\noindent
\begin{boxedminipage}{\textwidth}
We use extra code to prevent a subsection title from a previous section to permeate in a new section if there is no subsection in the beginning pages of the new section. Therefore we use the macro \verb|\rightheader| above for the right header, so that we can redefine this later. The criterion is whether there is a \verb|\section| command on the page that is not followed by a \verb|\subsection| command on the same page. This can't be checked with the marks that we currently have. So we define a new mark \texttt{which} that contains the level of the [sub]section command (0 = section, 1 = subsection). We redefine \verb|\sectionmark| and \verb\subsectionmark\ to put 0 and 1, resp. in this mark, and then we check whether the last one on the page is 0.

\begin{verbatim}
\extramarksnewmark{which}

\newcommand\checkrightreset{%
  \ifextramarksmissing{left}
  {}
  {% if there is a sectionmark on the page
    \ifnum\extramarkslast{which}=0
      % that is the last (no subsection following on the page)
      % then reset the subsection mark
      \extramarksreset{right}%
    \fi
  }%
}

\renewcommand{\sectionmark}[1]{%
  \extramarksleft{\thesection. #1}%
  \extramarksput{which}{0}%
  \renewcommand\rightheader{\checkrightreset\firstrightxmark}%
}

\renewcommand{\subsectionmark}[1]{%
  \extramarksright{\thesubsection\ #1}%
  \extramarksput{which}{1}%
}
\end{verbatim}
\end{boxedminipage}

\newpage
\pagestyle{fancy}

\section{Section One}

\subsection{Subsection One}

 \lipsum[1-2]

\section{Section Two}

 \lipsum[3]

\subsection{Subsection Two}

 \lipsum[4-7]

\section{Third section}
\label{sec:flaw}

{\bfseries In this section we see the reset of the ``inheritance'' of the subsection title if there is a section title, but no subsection title following it on the page. Therefore, because the following page doesn't have a subsection, it has the empty right header.}

\medskip

\lipsum

\subsection{Subsection from the third section}

{\bfseries This subsection ideally shouldn't end up in the header.}

\section{Fourth section}
\label{sec:missing}

\lipsum[1]

\subsection{Subsection from the fourth section}

{\bfseries However, this subsection ideally should end up in the header. Alas, this is not how it works.}

\medskip

\lipsum[1-4]

\subsection{A Fourth Subsection}

\lipsum[7-9]

\vspace{2cm}

\section{Fifth section}
\label{sec:push}

\lipsum[1-6]

\end{document}

https://tex.stackexchange.com/q/586066/113546
Answer:
https://tex.stackexchange.com/a/587094/113546
