\documentclass[oneside]{report}

\usepackage{fancyhdr}
\usepackage{lipsum}
\usepackage{fancyhdr}
\usepackage{blindtext}

\fancypagestyle{fancy}{%
  \renewcommand{\headrulewidth}{0.4pt}
  \fancyhf{}
  \fancyhead[L]{\leftmark}
  \fancyhead[R]{\rightmark}
  \fancyfoot[C]{\thepage}
}

\fancypagestyle{special}{%
  \fancyhf{}
  \renewcommand{\headrulewidth}{0pt}
  \fancyhead[L]{Special Page Style \nouppercase\leftmark}
  \fancyfoot[R]{\thepage}
}
\pagestyle{fancy}

\begin{document}

\chapter{Introduction}

This example shows the effects of having the header and footer definitions embedded in the page style. The page style \texttt{special} does have some relevant definitions inside it, and the page style \texttt{fancy} too. However, as the \texttt{special} page style changes the \verb|\headrulewidth|, page style \texttt{fancy} must also embed the default definition of this, which isn't logical.

\begin{verbatim}
\fancypagestyle{fancy}{%
  \renewcommand{\headrulewidth}{0.4pt}
  \fancyhf{}
  \fancyhead[L]{\leftmark}
  \fancyhead[R]{\rightmark}
  \fancyfoot[C]{\thepage}
}

\fancypagestyle{special}{%
  \fancyhf{}
  \renewcommand{\headrulewidth}{0pt}
  \fancyhead[L]{Special Page Style \nouppercase\leftmark}
  \fancyfoot[R]{\thepage}
}
\pagestyle{fancy}
\end{verbatim}

\bigskip

\lipsum[1-5]

\chapter{Special Chapter}
\pagestyle{special}
\lipsum

\chapter{Another Chapter}
\pagestyle{fancy}
\lipsum

\end{document}
