\documentclass{article}

\usepackage{lipsum}
\usepackage{boxedminipage}

\usepackage{fancyhdr}
\usepackage{extramarks}
\fancyhf{}
\fancyhead[L]{\firstleftxmark}
\newcommand\rightheader{\firstrightxmark}
\fancyhead[R]{\rightheader}
\fancyfoot[C]{\thepage}
\renewcommand*{\sectionmark}[1]{\extramarksleft{\thesection\ #1}}
\renewcommand*{\subsectionmark}[1]{\extramarksright{\thesubsection\ #1}}

\newcommand\rightreset{%
  \ifextramarksmissing{left}{}
  {% if there is a sectionmark on the page
    \ifextramarksmissing {right}
    % but no subsectionmark. then reset subsectionmark
    {\extramarksreset{right}}
    {}%
  }%
}

% Reset before the header only when the section was pushed to the next page
% The counter 'thispage' contains the page number where the \section
% command was processed, whereas 'page' is the page where the section
% command actually appears. If it was pushed to the following page, do
% the \rightreset (which will be before the actual mark).

\newcounter{thispage}
\newcommand\rightresetbefore{% 
  \ifnum \value{thispage}<\value{page}\rightreset\fi
}

\begin{document}
\pagestyle{plain}
\tableofcontents

\bigskip

\noindent
\begin{boxedminipage}{\textwidth}
This is example 35 in the \texttt{fancyhdr} documentation.

In this document we put the first section title (left) and the first subsection title (right) in the page headers. This is hard (probably impossible) to achieve with the standard \LaTeX{} marks.

We solve it with the new \texttt{extramarks.sty} (version 4.1 and later).

\begin{verbatim}
\usepackage{fancyhdr}
\usepackage{extramarks}
\fancyhf{}
\fancyhead[L]{\firstleftxmark}
\newcommand\rightheader{\firstrightxmark}
\fancyhead[R]{\rightheader}
\fancyfoot[C]{\thepage}
\renewcommand*{\sectionmark}[1]{%
    \extramarksleft{\thesection\ #1}}
\renewcommand*{\subsectionmark}[1]{%
    \extramarksright{\thesubsection\ #1}}
\end{verbatim}
\end{boxedminipage}

\begin{boxedminipage}{\textwidth}
In section~\ref{sec:missing} we use extra code to prevent a subsection title from a previous section to permeate in the new section if there is no subsection in the beginning pages of the new section. Therefore we have used the macro \verb|\rightheader| above for the right header, so that we can redefine this later.

\begin{verbatim}
\newcommand\rightreset{%
  \ifextramarksmissing{left}{}
  {% if there is a sectionmark on the page
    \ifextramarksmissing {right}
    % but no subsectionmark, then reset subsectionmark
    {\extramarksreset{right}}
    {}%
  }%
}

\newcounter{thispage}
\newcommand\rightresetbefore{% 
  \ifnum \value{thispage}<\value{page}\rightreset\fi
}
\end{verbatim}
In that section~\ref{sec:missing} we change \verb|\sectionmark| to communicate to the header whether the section starts in the upper half or the lower half of the page. See the \textsf{fancyhdr} documentation for the explanation.

\begin{verbatim}
\renewcommand{\sectionmark}[1]{%
  \extramarksleft{\thesection. #1}%
  \ifdim\pagetotal<0.5\pagegoal
    % put \rightreset before
    \renewcommand\rightheader{\rightreset\firstrightxmark}%
  \else
    % save the current page number
    \setcounter{thispage}{\value{page}}%
    % put \rightreset after,
    % but sometimes before (when page numbers differ)
    \renewcommand\rightheader{%
      \rightresetbefore\firstrightxmark\rightreset}%
  \fi
}
\end{verbatim}

The code that compares the page numbers can be seen in action in section~\ref{sec:six} on page~\pageref{sec:six}.

\end{boxedminipage}

\newpage
\pagestyle{fancy}

\section{Section One}

\subsection{Subsection One}

 \lipsum[1-2]

\section{Section Two}

 \lipsum[3]

\subsection{Subsection Two}

 \lipsum[4-7]

\section{Third section}

\textbf{This section ``inherits'' the subsection titles from the
  previous section (2.1), which is ugly. See the next page. It could
  even be a subsection title from several sections ago, see the next
  section.}

\medskip

\lipsum

\renewcommand{\sectionmark}[1]{%
  \extramarksleft{\thesection. #1}%
  \ifdim\pagetotal<0.5\pagegoal
    % put \rightreset before
    \renewcommand\rightheader{\rightreset\firstrightxmark}%
  \else
    % save the current page number
    \setcounter{thispage}{\value{page}}%
    % put \rightreset after,
    % but sometimes before (when page numbers differ)
    \renewcommand\rightheader{%
      \rightresetbefore\firstrightxmark\rightreset}%
  \fi
}

\section{Fourth section}
\label{sec:missing}

\textbf{In this section we reset the ``inheritance'' of the subsection title if there is a section title, but no subsection title on the page. Therefore this page will have an empty subsection title in the header.}

\begin{verbatim}
\newcommand\rightreset{%
  \ifextramarksmissing{left}{}
  {% if there is a sectionmark on the page
    \ifextramarksmissing {right}
    % but no subsectionmark, then reset subsectionmark
    {\extramarksreset{right}}
    {}%
  }%
}
\fancyheadinit{\rightreset}
\end{verbatim}

\medskip

\lipsum

\subsection{A Fourth Subsection}

\lipsum[8-11]

\section{Fifth section}

\textbf{This section starts at the lower half of the page. Therefore this page will inherit the subsection title in the header, but not on the next page.}

\medskip

\lipsum[1-7]

\subsection{Fifth subsection}

\lipsum[8-16]

\vspace{2cm}

\section{Sixth section}
\label{sec:six}

{\bfseries This section is processed at the lower half of  page~\thethispage. But as you can see it appears on page~\pageref{sec:six}. Therefore the code in \verb|\rightresetbefore| will come into action to do the reset before the header, because \mbox{thispage $<$ page}}

\medskip

\lipsum[1-6]

\end{document}

https://tex.stackexchange.com/q/586066/113546
Answer:
https://tex.stackexchange.com/a/587094/113546
