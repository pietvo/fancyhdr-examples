\documentclass{article}

\usepackage{lipsum}
\usepackage{boxedminipage}
\usepackage{ifthen}
\usepackage{afterpage}
\usepackage{fancyhdr}
\fancyhead{}
\fancyfoot[C]{\thepage}
\fancypagestyle*{intro}{%
  \renewcommand{\headrule}{}
  \fancyhead[R]{INTRODUCTION}
}
\fancypagestyle{contents}[intro]{\fancyhead[R]{CONTENTS}}

\fancypagestyle*{fancy}{%
  \fancyhead[L]{\FirstMark{2e-left}}
  \fancyhead[R]{\FirstMark{2e-right-nonempty}}
  \fancyfoot[C]{\thepage}
}
\renewcommand{\sectionmark}[1]{%
    \markboth{\thesection. #1}{}}
\renewcommand{\subsectionmark}[1]{%
    \markright{\thesubsection. #1}}

% This is for the second half of the document.

\NewMarkClass{section}
\NewMarkClass{subsection}
\NewMarkClass{which}
\InsertMark{which}{0}
\NewMarkClass{whichpage}
\newcommand\whichpage{}

\newcommand{\DEBUG}[1]{\message{#1}\ignorespaces} % Choose this for debugging
%\newcommand{\DEBUG}[1]{}             % Choose this for no debugging

\begin{document}
\pagestyle{intro}
\thispagestyle{contents}
\tableofcontents

\bigskip

\noindent
\begin{boxedminipage}{\textwidth}
This is example 35 in the \textsf{fancyhdr} documentation.

In this document we put the first section title (left) and the first subsection title (right) in the page headers. This is hard (probably impossible) to achieve with the traditional \LaTeX{} marks.

\textbf{NOTE: This requires a \LaTeX{} kernel that has the new marks mechanism (November 2022 or later).}

\medskip

This is the first attempt:

\begin{verbatim}
\usepackage{fancyhdr}
\fancypagestyle*{fancy}{%
  \fancyhead[L]{\FirstMark{2e-left}}
  \fancyhead[R]{\FirstMark{2e-right-nonempty}}
  \fancyfoot[C]{\thepage}
}
\pagestyle{fancy}
\renewcommand{\sectionmark}[1]{%
    \markboth{\thesection. #1}{}}
\renewcommand{\subsectionmark}[1]{%
    \markright{\thesubsection. #1}}
\end{verbatim}
\end{boxedminipage}

\noindent
\begin{boxedminipage}{\textwidth}
However, in section~\ref{sec:ugly_inheritance}, page~\pageref{sec:ugly_inheritance} we have the section title from that section~\ref{sec:ugly_inheritance}, but a subsection title from the previous section. This is ugly. In order to solve this we first change the code to use our own marks, so that we have more control. We need a mark for the section title, and one for the subsection title.

\medskip

Interim code:

\begin{verbatim}
\NewMarkClass{section}
\NewMarkClass{subsection}
\fancypagestyle*{fancy2}{%
  \fancyhead[L]{\FirstMark{section}}
  \fancyhead[R]{\FirstMark{subsection}}
}
\pagestyle{fancy2}
\renewcommand{\sectionmark}[1]{%
    \InsertMark{section}{\thesection. #1}}
\renewcommand{\subsectionmark}[1]{%
    \InsertMark{subsection}{\thesubsection. #1}}
\end{verbatim}

Now we have to check whether there is a \verb|\section| command on the page that is not followed by a \verb|\subsection| command on the same page. This can't be checked with the marks that we currently have. So we define an additional mark `\texttt{which}' that contains the level of the last [sub]section command. We redefine \verb|\sectionmark| and \verb\subsectionmark\ to put ``0'' and ``1'', resp. in this mark, and then we check in the right header whether there is a subsection title on the page, and if the the last value of `\texttt{which}' on the page is ``0'' or ``1''.

\begin{verbatim}
\NewMarkClass{which}
\InsertMark{which}{0}% Initialize

\renewcommand{\subsectionmark}[1]{%
  \InsertMark{subsection}{\thesubsection. #1}%
  \InsertMark{which}{1}%
}

% \renewcommand{\sectionmark}, see below
\end{verbatim}
\end{boxedminipage}

\noindent
\begin{boxedminipage}{\textwidth}

In section~\ref{sec:missing} we use the extra code to check this `\texttt{which}' mark to select the proper subsection title.

In that section we also communicate to the header whether the section starts at or very near to the top of the page.

We also check whether the the section was pushed to the next page.
The mark \texttt{whichpage} contains the page number where the \verb|\section|
command was processed, whereas \verb|\thepage| is the page where the section
command actually appears. If it was pushed to a following page, these two are different.
Both these cases are stored in the variable \verb|\whichpage|, as a value that differs from \verb|\thepage| of the page where the section title appears.

See the \textsf{fancyhdr} documentation for the explanation.

\MakeLinkTarget{}\label{good_code}
\begin{verbatim}
\NewMarkClass{whichpage}

\AddToHook{cmd/section/before}{%
  \ifthenelse{\lengthtest{\pagetotal<4\baselineskip}}{%
    % Section header near top of page
    \renewcommand{\whichpage}{top\thepage}%
  }
  {\renewcommand{\whichpage}{\thepage}}%
}

\renewcommand{\sectionmark}[1]{%
  \InsertMark{section}{\thesection\ #1}%
  \InsertMark{which}{0}%
  \InsertMark{whichpage}{\whichpage}%
}
\end{verbatim}

The code that compares the page numbers can be seen in action in section~\ref{sec:push} on page~\pageref{sec:push}.

\end{boxedminipage}

\newpage
\pagestyle{fancy}

\section{Section One}

\subsection{Subsection One}

\textbf{NOTE: Sections 1--3 use the simple version of the headers, which gives an `ugly inheritance' on page~\pageref{ugly_page}.}

\smallskip

 \lipsum[1-2]

\section{Section Two}

 \lipsum[3]

\subsection{Subsection Two}
\label{sec:inherited}

 \lipsum[4-7]

\section{Third section}
\label{sec:ugly_inheritance}

\textbf{This section ``inherits'' the subsection title from the
  previous section (\ref{sec:inherited}), which is good on this page. But it should not be used on the next page.
  It could even be a subsection title from several sections ago. We will repair this in the next section.}
\afterpage{\MakeLinkTarget{}\label{ugly_page}}

\medskip

\lipsum

\fancypagestyle*{fancy2}{%
  \fancyhead[L]{\FirstMark{section}}
  \fancyhead[R]{%
    \IfMarksEqualTF{subsection}{top}{first}
      {% no subsection mark on this page
        \DEBUG{Page \thepage, no subsection mark on this page^^J}%
        \ifthenelse{%
          % previous page ended in a subsection
          \TopMark{which}=1\and
                 % Is there a section on the page?
          \equal{\IfMarksEqualTF{section}{top}{first}
                {\thepage}{\FirstMark{whichpage}}}{\thepage}}%
          {
            \DEBUG{Page \thepage, previous page ended in a subsection^^J
                 and whichpage = \FirstMark{whichpage}}% 
            % use previous subsection unless suppressed by section on top
            \TopMark{subsection}% = \FirstMark{subsection}%
          }
        % otherwise use empty right header
        {\DEBUG{Page \thepage, empty right header^^J}}%
      }
      {% if there is a subsectionmark on the page, use it
        \DEBUG{Page \thepage, there is a subsection mark on this page^^J}%
        \FirstMark{subsection}%
      }%
  }
}
\pagestyle{fancy2}

% The counter 'thispage' contains the page number where the \section
% command was processed, whereas 'page' is the page where the section
% command actually appears.

\newcounter{thispage}% Only for the documentation

\AddToHook{cmd/section/before}{%
  \DEBUG{Section \the\numexpr\value{section}+1\relax\space is processed on page \thepage,^^J
    position \the\pagetotal.^^J}%
  \ifthenelse{\lengthtest{\pagetotal<4\baselineskip}}{%
    \DEBUG{Section header near top of page^^J}%
    \renewcommand{\whichpage}{top\thepage}%
  }%
  {\renewcommand{\whichpage}{\thepage}}%
  \setcounter{thispage}{\thepage}% Only for the documentation
}

\renewcommand{\sectionmark}[1]{%
  \InsertMark{section}{\thesection. #1}%
  \InsertMark{which}{0}%
  \InsertMark{whichpage}{\whichpage}%
}

\renewcommand{\subsectionmark}[1]{%
  \InsertMark{subsection}{\thesubsection. #1}%
  \InsertMark{which}{1}%
}

\section{Fourth section}
\label{sec:missing}

From this page on we suppress the ``inheritance'' of a subsection title if
  \begin{itemize}
  \item the previous page ended with a section, not a subsection,
  \item or the current page has a section title at or near the top.
  \end{itemize}
On the current page there is no subsection, and the previous page ended on a section.
Therefore this page will have an empty subsection title in the right header.

\bigskip

Besides the code on page~\pageref{good_code} we have the following definition of the headers (NOTE: the actual code also contains debugging commands). For explanation, see Example 35 in the \textsf{fancyhdr} documentation, in the section ``I want the first section and the first subsection in my headers''.

\begin{verbatim}
\fancypagestyle*{fancy2}{%
  \fancyhead[L]{\FirstMark{section}}

  \fancyhead[R]{%
    \IfMarksEqualTF{subsection}{top}{first}
      {% no subsection mark on this page
        \ifthenelse{%
          % previous page ended in a subsection
          \TopMark{which}=1\and
                 % Is there a section on the page?
          \equal{\IfMarksEqualTF{section}{top}{first}
                {\thepage}{\FirstMark{whichpage}}}{\thepage}}%
          {
            % use previous subsection unless suppressed by section on top
            \TopMark{subsection}%
          }
        % otherwise use empty right header
        {}%
      }
      {% if there is a subsectionmark on the page, use it
        \FirstMark{subsection}%
      }%
  }
}

\pagestyle{fancy2}
\end{verbatim}

The suppression of the inherited right header can be seen on the pages \pageref{sec:push} and \pageref{sec:newpage}.

\medskip

\lipsum[1-3]

\subsection{A Fourth Subsection}

\lipsum[8-11]

\section{Fifth section}

\lipsum[1-7]

\subsection{Fifth subsection}

\lipsum[8-17]

\section{Sixth section}
\label{sec:push}

{\bfseries This section is processed at the bottom of page~\thethispage. But as you can see it has been pushed to page~\pageref{sec:push}.}\\
Therefore the code that checks if \verb|\FirstMark{whichpage}| is equal to \verb|\thepage| suppresses the right header
(\verb|\FirstMark{whichpage}| = \thethispage{} and  \verb|\thepage| = \pageref{sec:push}).

\medskip

\lipsum[1-6]

\subsection{Sixth subsection}

\lipsum[7]

\newpage
\section{Seventh section after a page break}
\label{sec:newpage}

Because this section starts at the top of the page, and there is no subsection on this page, the right header is also empty.

\medskip

\lipsum[8-10]

\end{document}

https://tex.stackexchange.com/q/586066/113546
Answer:
https://tex.stackexchange.com/a/587094/113546
