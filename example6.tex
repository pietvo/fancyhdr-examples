\documentclass{book}
\usepackage{lipsum}
\usepackage{makeidx}
\usepackage{boxedminipage}
\usepackage{fancyhdr}
\pagestyle{fancy}
\fancyhead[LE,RO]{\rightmark}
\fancyhead[LO,RE]{\leftmark}
\renewcommand{\chaptermark}[1]{%
  \markboth{\chaptername\ \thechapter.\ #1}{}}
\renewcommand{\sectionmark}[1]{\markright{\thesection.\ #1}}
\makeindex

\begin{document}
\chapter{Introduction}

\index{lipsum}
\begin{boxedminipage}{\textwidth}
This is example 6 in the fancyhdr documentation. 
This demonstrates the use of \verb|\chaptermark| and \verb|\sectionmark|. In contrast to the standard \LaTeX{} definitions, we don't use \verb|\MakeUppercase|.
\begin{verbatim}
\fancyhead[LE,RO]{\rightmark}
\fancyhead[LO,RE]{\leftmark}
\renewcommand{\chaptermark}[1]{%
  \markboth{\chaptername\ \thechapter.\ #1}{}}
\renewcommand{\sectionmark}[1]{\markright{\thesection.\ #1}}
\end{verbatim}
On the Index page the header will still be ``INDEX'' in uppercase because the \texttt{book} class uses \verb|\MakeUppercase| for the  \verb|\chaptermark| in Index, Bibliography, etc.
\end{boxedminipage}

\section{The Problem}
\label{sec:problem}

\lipsum[1]

\section{Evaluation}

\lipsum[2]

\chapter{Implementation}

Introduction of this chapter

\lipsum[3]

\section{First steps}
\label{sec:first-steps}

\lipsum
\printindex
\newpage
``INDEX'' should be in header in uppercase.
\end{document}
