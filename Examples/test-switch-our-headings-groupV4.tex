\documentclass[openany]{book}
\usepackage{lipsum}
\usepackage{boxedminipage}
\usepackage[headings]{fancyhdr}
\pagestyle{fancy}
\renewcommand{\chaptermark}[1]{\markboth{\thechapter.\ #1}{}}
\renewcommand{\sectionmark}[1]{\markright{\thesection.\ #1}}

\fancypagestyle{myfancy}{
  \renewcommand{\chaptermark}[1]{\markboth{\thechapter.
      \MakeUppercase{##1}}{}}
  \renewcommand{\sectionmark}[1]{\markright{\thesection.
      \MakeUppercase{##1}}}
}

\fancypagestyle{myotherfancy}{
  \renewcommand{\chaptermark}[1]{\markboth{\thechapter.\ ##1}{}}
  \renewcommand{\sectionmark}[1]{\markright{\thesection.\ ##1}}
  \fancyhf{}
  \fancyhead[LE]{\textbf{\rightmark}}
  \fancyhead[RO]{\textbf{\leftmark}}
  \fancyfoot[C]{\textbf{\thepage}}
}

\begin{document}

\tableofcontents

\newpage
\thispagestyle{plain}

\noindent
\begin{boxedminipage}{\textwidth}
  This is test-switch-our-headings-groupV4.tex

It demonstrates switching between page styles based on
\texttt{fancyhdr} in the \texttt{book} class. We are switching between
four page styles:\\ the \texttt{fancyhdr}-based page style
\texttt{headings} (package option \texttt{headings}), our page style
\texttt{fancy}, and two \texttt{fancyhdr}-based styles \texttt{myfancy}
and \texttt{myotherfancy}, defined with \verb|\fancypagestyle|. And
chapter opening pages use page style \texttt{plain}.
  Here are the definitions:

\begin{verbatim}
\pagestyle{fancy}
\renewcommand{\chaptermark}[1]{\markboth{\thechapter.\ #1}{}}
\renewcommand{\sectionmark}[1]{\markright{\thesection.\ #1}}

\fancypagestyle{myfancy}{
  \renewcommand{\chaptermark}[1]{\markboth{\thechapter.
      \MakeUppercase{##1}}{}}
  \renewcommand{\sectionmark}[1]{\markright{\thesection.
      \MakeUppercase{##1}}}
}

\fancypagestyle{myotherfancy}{
  \renewcommand{\chaptermark}[1]{\markboth{\thechapter.\ ##1}{}}
  \renewcommand{\sectionmark}[1]{\markright{\thesection.\ ##1}}
  \fancyhf{}
  \fancyhead[LE]{\textbf{\rightmark}}
  \fancyhead[RO]{\textbf{\leftmark}}
  \fancyfoot[C]{\textbf{\thepage}}
}
\end{verbatim}
\end{boxedminipage}

\chapter{Introduction}

\noindent
\begin{boxedminipage}{\textwidth}
We start with page style \texttt{fancy} with the following headers:
\begin{verbatim}
  \renewcommand{\chaptermark}[1]{\markboth{\thechapter.\ ##1}{}}
  \renewcommand{\sectionmark}[1]{\markright{\thesection.\ ##1}}
\end{verbatim}
%
Chapter pages: just page style \texttt{plain}.
\\
Even page, left:
\begin{quote}
  \textsl{1.1. The Problem} (= section number. section title)
\end{quote}
Even page, right:
\begin{quote}
  \textsl{1. Introduction} (= chapter number. chapter title)
\end{quote}
On odd pages it is mirrored, probably with a different section.

These come out as \textsl{slanted}, because that is the default for page style \texttt{fancy}.
The page number comes on the footer.
\end{boxedminipage}
\bigskip

\lipsum[1]
\newpage
\section{The Problem}
\label{sec:problem}

\lipsum[2-3]

\section{Evaluation}

\lipsum[3-5]

\lipsum[6]

\newpage
\begingroup
\pagestyle{headings}
\chapter{Second chapter}

\noindent
\begin{boxedminipage}{\textwidth}
In this chapter we have switched to the \texttt{fancyhdr} version of page style \texttt{headings} \textbf{INSIDE A GROUP}. The marks are different than in page style fancy. For example, they are in uppercase.

{\bfseries Please note that because page style \texttt{headings} uses a different \verb|\chaptermark| than page style \texttt{fancy}, the switching of the page style must be done before the \verb|\chapter| command, and it must be preceded by a \verb|\newpage|, and this is also necessary in the following chapters.}
\begin{verbatim}
\newpage
\pagestyle{headings}
\chapter{Second chapter}
\end{verbatim}

The headers should come out as:
\\
Chapter pages: just page style \texttt{plain}.
\\
Right header on even page:
\begin{quote}
  \textsl{CHAPTER 2. SECOND CHAPTER}
\end{quote}
Left header on odd page:
\begin{quote}
  \textsl{2.1. ANOTHER SECTION}
\end{quote}
These come out as \textsl{slanted}, because that is the default for page style \texttt{fancy}.
The page number comes on the other side of the header, not slanted.
\end{boxedminipage}
\bigskip

\lipsum[1]

\section{Another section}

\lipsum

\newpage
\endgroup
\pagestyle{fancy}
\chapter{Third chapter}

\noindent
\begin{boxedminipage}{\textwidth}
Here we switch back to page style \texttt{fancy}. Because the previous page style was inside a group we get our own chapter and section marks and header layout back.

The headers should come out as:
\\
Chapter pages: just page style \texttt{plain}.
\\
Even page, left:
\begin{quote}
  \textsl{3.1. YA section}
\end{quote}
Even page, right:
\begin{quote}
  \textsl{3. Third chapter}
\end{quote}
On odd pages it is mirrored, probably with a different section.

These come out as \textsl{slanted}, because that is the default for page style \texttt{fancy}.
The page number comes on the footer.
\end{boxedminipage}
\bigskip

\lipsum[1]

\section{YA section}

\lipsum

\newpage
\pagestyle{myfancy}
\chapter{Fourth chapter}

\noindent
\begin{boxedminipage}{\textwidth}
Here we switch to page style \texttt{myfancy}.
Now we see that we get the marks as defined in the definition of page style \texttt{myfancy}.
This is because the definitions of the marks are  part of that definition.
\begin{verbatim}
\fancypagestyle{myfancy}{
  \renewcommand{\chaptermark}[1]{\markboth{\thechapter.
      \MakeUppercase{##1}}{}}
  \renewcommand{\sectionmark}[1]{\markright{\thesection.
      \MakeUppercase{##1}}}
}
\end{verbatim}

However, the layout of the headers and footers is not defined in page style \texttt{myfancy}. Therefore we get the layout as in the previous page style i.e.\ page style \texttt{fancy}. 
But because of our own marks, the contents of the headers are in uppercase, and the word 'CHAPTER' is not present.

The headers should come out as:
\\
Chapter pages: just page style \texttt{plain}.
\\
Even page, left:
\begin{quote}
  \textsl{4.1. YANO SECTION}
\end{quote}
Even page, right:
\begin{quote}
  \textsl{4. FOURTH CHAPTER}
\end{quote}
On odd pages it is mirrored.
These come out as \textsl{slanted}, because that is the default for page style \texttt{fancy}.
The page number comes on the footer.

Please note that the definition of \texttt{myfancy} does not include \verb|\fancyhead|. So it picks up whatever is current when it is invoked. It is generally better to include these definition, however.
\end{boxedminipage}
\bigskip

\lipsum[1]

\section{YAno section}

\lipsum

\newpage
\pagestyle{myotherfancy}
\chapter{Fifth chapter}

\noindent
\begin{boxedminipage}{\textwidth}
Here we switch to page style \texttt{myotherfancy}.
Now we see that we get the marks as defined in the definition of page style \texttt{myotherfancy}.
This is because the definitions of the marks are  part of that definition.
\begin{verbatim}
\fancypagestyle{myotherfancy}{
  \renewcommand{\chaptermark}[1]{\markboth{\thechapter.\ ##1}{}}
  \renewcommand{\sectionmark}[1]{\markright{\thesection.\ ##1}}
  \fancyhf{}
  \fancyhead[LE]{\textbf{\rightmark}}
  \fancyhead[RO]{\textbf{\leftmark}}
  \fancyfoot[C]{\textbf{\thepage}}
}
\end{verbatim}
And also, the layout of the headers and footers is  defined in this page style. It defines the headers and footers \textbf{bold} instead of \textsl{slanted}.
The contents of the headers are not in uppercase, and the word 'CHAPTER' is not present.

The headers should come out as:
\\
Chapter pages: just page style \texttt{plain}.
\\
Right header on odd page:
\begin{quote}
  \textbf{5. Fifth chapter}
\end{quote}
Left header on even page:
\begin{quote}
  \textbf{5.1. YAno section}
\end{quote}
The page number comes on the footer.
These come out as \textbf{bold}.
\end{boxedminipage}
\bigskip

\lipsum[1]

\section{YAno section}

\lipsum

\end{document}
