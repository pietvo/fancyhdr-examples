\documentclass[twoside]{article}

\usepackage{lipsum}
\usepackage{boxedminipage}
\usepackage{fancyhdr}
\pagestyle{fancy}
\fancyhead[L]{LEFT HEADER}
\fancyhead[R]{RIGHT HEADER}
\fancyfoot[L]{LEFT FOOTER}
\fancyfoot[R]{RIGHT FOOTER}
\addtolength{\headwidth}{1cm}

\begin{document}
\noindent
\begin{boxedminipage}{\textwidth}
This is a test for various ways to change and use \verb|\headwidth| for both headers and footer.

On the first two pages we enlarge the \verb|\headwidth| with 1~cm. 
\begin{verbatim}
\addtolength{\headwidth}{1cm}
\end{verbatim}
As this is a two-sided document, the header and footer will stick out on the right for odd pages and on the left for even pages.

\end{boxedminipage}

\bigskip

\lipsum[1-9]

\newpage
\fancyheadoffset[LR]{1cm}

\noindent
\begin{boxedminipage}{\textwidth}
On the following pages we enlarge the headers with
\begin{verbatim}
\fancyheadoffset[LR]{1cm}
\end{verbatim}
So  the header and footer will stick out on both sides for 1~cm on all pages.
As the footer doesn't have offsets defined, its width reverts to the original \verb|\textwidth|.
So the enlarged \verb|\headwidth| from the beginning of the document will not be used once a \verb|\fancy...offset| is given.
\end{boxedminipage}
 
\bigskip

\lipsum
\end{document}
