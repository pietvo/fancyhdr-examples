\documentclass{article}

\usepackage{lipsum}
\usepackage{boxedminipage}
\usepackage{ifthen}
\usepackage{fancyhdr}
\fancyhead{}
\fancyfoot[C]{\thepage}
\fancypagestyle*{intro}{%
  \renewcommand{\headrule}{}
  \fancyhead[R]{INTRODUCTION}
}
\fancypagestyle{contents}[intro]{\fancyhead[R]{CONTENTS}}

\NewMarkClass{section}
\NewMarkClass{subsection}
\NewMarkClass{which}
\InsertMark{which}{0}
\NewMarkClass{whichpage}
\newcommand\whichpage{}

\newcommand{\DEBUG}[1]{\message{#1}\ignorespaces} % Choose this for debugging
%\newcommand{\DEBUG}[1]{}             % Choose this for no debugging

\fancypagestyle*{fancy}{%
  \fancyhead[L]{\FirstMark{section}}
  \fancyhead[R]{%
    \IfMarksEqualTF{subsection}{top}{first}
      {% no subsection mark on this page
        \DEBUG{Page \thepage, no subsection mark on this page^^J}%
        \ifthenelse{%
          % previous page ended in a subsection
          \TopMark{which}=1\and
                 % Is there a section on the page?
          \equal{\IfMarksEqualTF{section}{top}{first}
                {\thepage}{\FirstMark{whichpage}}}{\thepage}}%
          {
            \DEBUG{Page \thepage, previous page ended in a subsection^^J
                 and whichpage = \FirstMark{whichpage}}% 
            % use previous subsection unless suppressed by section on top
            \TopMark{subsection}% = \FirstMark{subsection}%
          }
        % otherwise use empty right header
        {\DEBUG{Page \thepage, empty right header^^J}}%
      }
      {% if there is a subsectionmark on the page, use it
        \DEBUG{Page \thepage, there is a subsection mark on this page^^J}%
        \FirstMark{subsection}%
      }%
  }
}

\newcounter{thispage}

\AddToHook{cmd/section/before}{%
  \DEBUG{Section \the\numexpr\value{section}+1\relax\space is processed on page \thepage,^^J
    position \the\pagetotal.^^J}%
  \ifthenelse{\lengthtest{\pagetotal<4\baselineskip}}{%
    \DEBUG{Section header near top of page^^J}%
    \renewcommand{\whichpage}{top\thepage}%
  }%
  {\renewcommand{\whichpage}{\thepage}}%
  \setcounter{thispage}{\thepage}% Only for the documentation
}

\renewcommand{\sectionmark}[1]{%
  \InsertMark{section}{\thesection. #1}%
  \InsertMark{which}{0}%
  \InsertMark{whichpage}{\whichpage}%
}

\renewcommand{\subsectionmark}[1]{%
  \InsertMark{subsection}{\thesubsection. #1}%
  \InsertMark{which}{1}%
}

\begin{document}
\pagestyle{intro}
\thispagestyle{contents}
\tableofcontents

\bigskip

\noindent
\begin{boxedminipage}{\textwidth}
This is example 35+floatpage, a variant of example 35b in the \texttt{fancyhdr} documentation.

{\em In this variant we have the final code of example 35 directly in the preamble, and we added a float page, to show that a section title can be pushed across more than one page boundary, and that this will be correctly detected.}

In this document we put the first section title (left) and the first subsection title (right) in the page headers. This is hard (probably impossible) to achieve with the standard \LaTeX{} marks.

\textbf{NOTE: This requires a \LaTeX{} kernel that has the new marks mechanism (November 2022 or later).}

We define two marks: \texttt{section} and \texttt{subsection} for the section and subsection titles, resp.

We also have to check whether there is a \verb|\section| command on the page that is not followed by a \verb|\subsection| command on the same page. This can't be checked with the standard \LaTeX{} marks. So we define an additional mark `\texttt{which}' that contains the level of the last [sub]section command. We redefine \verb|\sectionmark| and \verb\subsectionmark\ to put ``0'' and ``1'', resp. in this mark, and then we check in the right header whether there is a subsection title on the page, and if the the last value of `\texttt{which}' on the page is ``0'' or ``1''.

\begin{verbatim}
\NewMarkClass{section}
\NewMarkClass{subsection}
\NewMarkClass{which}
\InsertMark{which}{0}
\end{verbatim}
\end{boxedminipage}

\noindent
\begin{boxedminipage}{\textwidth}
We change \verb|\sectionmark| to communicate to the header whether the section starts at or near the top of the page.

We also check whether the the section was pushed to the next page.
The mark \texttt{whichpage} contains the page number where the \verb|\section|
command was processed, whereas \verb|\thepage| is the page where the section
command actually appears. If it was pushed to a following page, these two are different.
Both these cases are stored in the variable \verb|\whichpage|, as a value that differs from \verb|\thepage| of the page where the section title appears.

See the \textsf{fancyhdr} documentation for the explanation.

\begin{verbatim}
\NewMarkClass{whichpage}
\newcommand\whichpage{}

\AddToHook{cmd/section/before}{%
  \ifthenelse{\lengthtest{\pagetotal<4\baselineskip}}{%
    % Section header near top of page
    \renewcommand{\whichpage}{top\thepage}%
  }
  {\renewcommand{\whichpage}{\thepage}}%
}

\renewcommand{\sectionmark}[1]{%
  \InsertMark{section}{\thesection. #1}%
  \InsertMark{which}{0}%
  \InsertMark{whichpage}{\whichpage}%
}

\renewcommand{\subsectionmark}[1]{%
  \InsertMark{subsection}{\thesubsection. #1}%
  \InsertMark{which}{1}%
}
\end{verbatim}
\end{boxedminipage}

\newpage
\noindent
\begin{boxedminipage}{\textwidth}
Then in the right header we check if \texttt{whichpage} is equal to the actual page number \verb|\thepage|. The complete algorithm for the right header is:\\[1ex]

\textbf{if} there is no subsection on the page\\
\textbf{then}\\
\hspace*{2em}\textbf{if} the previous page ended in a subsection\\
\hspace*{3em}\textbf{and} there is no section title at or near the top of the page\\
\hspace*{2em}\textbf{then} use (inherit) that subsection\\
\hspace*{2em}\textbf{else} use an empty header\\
\hspace*{2em}\textbf{fi}\\
\textbf{else} (there is at least one subsection on the page)\\
\hspace*{2em}use the first subsection\\
\textbf{fi}\\[1ex]

The test for ``there is no section title at or near the top of the page'' is by comparing \texttt{whichpage} to \verb|\thepage| and checking if there actually is a section title on the page.

\begin{verbatim}
\fancypagestyle*{fancy}{%
  \fancyhead[L]{\FirstMark{section}}
  \fancyhead[R]{%
    \IfMarksEqualTF{subsection}{top}{first}
      {% no subsection mark on this page
        \ifthenelse{%
          % previous page ended in a subsection
          \TopMark{which}=1\and
                 % Is there a section on the page?
          \equal{\IfMarksEqualTF{section}{top}{first}
                {\thepage}{\FirstMark{whichpage}}}{\thepage}}%
          {
            % use previous subsection unless suppressed
            % by section on top
            \TopMark{subsection}%
          }
        % otherwise use empty right header
        {}}%
      }
      {% if there is a subsectionmark on the page, use it
        \FirstMark{subsection}%
      }%
  }
}
\end{verbatim}

The code that compares the page numbers can be seen in action in section~\ref{sec:push} on page~\pageref{sec:push} and section~\ref{sec:newpage} on page~ \pageref{sec:newpage}.
\end{boxedminipage}

\newpage
\pagestyle{fancy}

\section{Section One}

\subsection{Subsection One}

 \lipsum[1-2]

\section{Section Two}

 \lipsum[3]

\subsection{Subsection Two}
\label{sec:inherited}

 \lipsum[4-7]

\section{Third section}
\label{sec:flaw}

\textbf{This section ``inherits'' the subsection title from the
  previous section (\ref{sec:inherited}), which is good on this page. But it is suppressed on the next page.}

\medskip

\lipsum

\section{Fourth section}
\label{sec:missing}

{\bfseries This page has no subsection, and the previous page ended in a section (not a subsection). Therefore this page will have an empty right header.}

\medskip

\lipsum[1-4]

\subsection{A Fourth Subsection}

\lipsum[7-9]

\section{Fifth section}

\lipsum[1-7]

\subsection{Fifth subsection}

\lipsum[8-16]

\begin{figure}[p]
  \centering
  \vspace{0.4\textheight}
  \caption{A figure}
\end{figure}

\begin{table}[p]
  \centering
  \vspace{0.5\textheight}
  \caption{A table}
\end{table}

\section{Sixth section}
\label{sec:push}

{\bfseries This section is processed at the bottom of page~\thethispage. But as you can see it has been pushed to page~\pageref{sec:push}.}\\
Therefore the code that checks if \verb|\FirstMark{whichpage}| is equal to \verb|\thepage| suppresses the right header
(\verb|\FirstMark{whichpage}| = \thethispage{} and  \verb|\thepage| = \pageref{sec:push}).

\medskip

\lipsum[1-6]

\subsection{Sixth subsection}

\lipsum[7]

\newpage
\section{Seventh section after a page break}
\label{sec:newpage}

Because this section starts at the top of the page, and there is no subsection on this page, the right header is also empty.

\medskip

\lipsum[8-10]

\end{document}

https://tex.stackexchange.com/q/586066/113546
Answer:
https://tex.stackexchange.com/a/587094/113546
