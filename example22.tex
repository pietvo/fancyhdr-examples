\documentclass[openany]{book}
\usepackage{lipsum}
\usepackage{ifthen}
\usepackage{lastpage}
\usepackage{boxedminipage}
\usepackage{fancyhdr}
\pagestyle{fancy}
\fancyfoot[C]{\thepage\ of \pageref{LastPage}}
\fancyfoot[R]{\ifthenelse{\isodd{\value{page}} \and \not
     \( \value{page}=\pageref{LastPage} \) }{please turn over}{}}
\fancypagestyle{plain}{\fancyhead{}\renewcommand{\headrule}{}}
\begin{document}

\tableofcontents
\bigskip

\noindent
\begin{boxedminipage}{\textwidth}
This is example 22 in the fancyhdr documentation.

We use the page number style ``n of m'' where ``m'' is the number of pages. This is done using the \texttt{lastpage} package. As we want the same style on chapter opening pages, and others that use the \texttt{plain} page style, we redefine the \texttt{plain} page style with \verb|\fancypagestyle|.

To get all this working you usually have to do one additional \LaTeX{} run.

\begin{verbatim}
\usepackage{lastpage}
 . . .
\fancyfoot[C]{\thepage\ of \pageref{LastPage}}
\fancypagestyle{plain}{\fancyhead{}\renewcommand{\headrule}{}}
\end{verbatim}

This example also puts  ``Please turn over'' in the righthand side footer on odd pages, except the last page. This is done with the following code:

\begin{verbatim}
\usepackage{ifthen}
 . . . 
\fancyfoot[R]{\ifthenelse{\isodd{\value{page}} \and \not
     \( \value{page}=\pageref{LastPage} \) }{please turn over}{}}
\end{verbatim}
\end{boxedminipage}

\chapter{Introduction}

\lipsum

\section{The Problem}
\label{sec:problem}

\lipsum[1]

\section{Evaluation}

\lipsum

Some more text.

Some more text.

Some more text.

Some more text.

\end{document}
