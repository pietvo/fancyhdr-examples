\documentclass{article}
\usepackage[headheight=40pt, includehead]{geometry} % for margins on a A4paper

\usepackage{fancyhdr}
\usepackage{graphicx}
\usepackage{kantlipsum}

\pagestyle{fancy}
\fancyhf{}
\fancyhead[L]{\sffamily
    \fancyhdrbox{\includegraphics[height=3\normalbaselineskip]{example-image}}}
\fancyhead[R]{%
  \fancyhdrbox[][4cm]{Our Office \\ Street 1 \\ City1 }%
  \fancyhdrbox[l]{ Our Factory \\ Street 2 \\ City2 }
}

\setlength{\parindent}{0pt}
\setlength{\parskip}{4pt}

\begin{document}

This example shows the use of \verb|\fancyhdrbox| in the header. The left header has an image in a \verb|\fancyhdrbox| with the default alignment. The right header has two  \verb|\fancyhdrbox|es, one with an explicit width of 4cm, the second one with its natural width.

\textbf{Header definition:}

\begin{verbatim}
\pagestyle{fancy}
\fancyhf{}
\fancyhead[L]{\sffamily
    \fancyhdrbox{\includegraphics[height=3\normalbaselineskip]{example-image}}}
\fancyhead[R]{%
  \fancyhdrbox[][4cm]{Our Office \\ Street 1 \\ City1 }%
  \fancyhdrbox[l]{ Our Factory \\ Street 2 \\ City2 }
}
\end{verbatim}

\kant[1-2]

\end{document}
