\documentclass{article}

\usepackage{graphicx}
\usepackage{afterpage}
\usepackage{lipsum}
\usepackage{boxedminipage}
\usepackage{subcaption}

\newcommand{\cs}[1]{\texttt{\char`\\#1}}
\renewcommand{\thesection}{\Alph{section}}

\begin{document}
\tableofcontents
\medskip
\listoffigures
\medskip

\section*{Introduction}
\addcontentsline{toc}{section}{Introduction}

\newcommand{\fakecaption}[2]{% #1 = figure label #2 = caption
  \par Figure~\ref{#1}: #2
}
\newcommand{\myfigures}[3]{% #1 = position #2 = label #3 = caption
\begin{figure}[#1]
  \centering
  \includegraphics[scale=0.5]{example-image-a}
  \caption[#3] % For the list of figures
          {#3 (a)}
  \label{#2}
\end{figure}
\begin{figure}[#1]
  \centering
  \includegraphics[scale=0.5]{example-image-b}
  \fakecaption{#2}{#3 (b)}
\end{figure}
\begin{figure}[#1]
  \centering
  \includegraphics[scale=0.5]{example-image-c}
  \fakecaption{#2}{#3 (c)}
\end{figure}
\begin{figure}[#1]
  \centering
  \includegraphics[scale=0.5]{example-image}
  \fakecaption{#2}{#3 (d)}
\end{figure}
}

\noindent
\begin{boxedminipage}{\textwidth}
This is example 33 in the \texttt{fancyhdr} documentation.

It shows some ways to split a figure over two or more pages. We do this by putting the pieces in separate \texttt{figure} environments. The first part has a real caption, which will appear in the list of figures. The other parts have fake captions.

This example is just about figure placement; it has no \texttt{fancyhdr} code.

We put the figures in a macro \texttt{myfigures} and then include it in various ways.

\begin{verbatim}
\newcommand{\fakecaption}[2]{% #1 = figure label #2 = caption
  \par Figure~\ref{#1}: #2
}
\newcommand{\myfigures}[3]{% #1 = position #2 = label #3 = caption
\begin{figure}[#1]
  \centering
  \includegraphics[scale=0.5]{example-image-a}
  \caption[#3] % For the list of figures
          {#3 (a)}
  \label{#2}
\end{figure}
\begin{figure}[#1]
  \centering
  \includegraphics[scale=0.5]{example-image-b}
  \fakecaption{#2}{#3 (b)}
\end{figure}
\begin{figure}[#1]
  \centering
  \includegraphics[scale=0.5]{example-image-c}
  \fakecaption{#2}{#3 (c)}
\end{figure}
\begin{figure}[#1]
  \centering
  \includegraphics[scale=0.5]{example-image}
  \fakecaption{#2}{#3 (d)}
\end{figure}
}
\end{verbatim}
\end{boxedminipage}

% A
\section{Split into separate \texttt{figure} environments}

\begin{boxedminipage}{\textwidth}
\begin{verbatim}

\myfigures{!htbp}{fig:first}{This is a multipart figure}

\end{verbatim}
\end{boxedminipage}

\medskip

\myfigures{!htbp}{fig:first}{This is a multipart figure}

\lipsum[1-2]

% B
\section{Use \cs{afterpage} with \cs{clearpage}}

\begin{boxedminipage}{\textwidth}
The \cs{clearpage} will make sure that other floats  are output first, and therefore cannot intrude between our figures.
We put it in \cs{afterpage}, otherwise we might have a half-filled page.
But if there is enough space left on a page, some of the text will go between the figures.

\begin{verbatim}
\afterpage{\clearpage
           \myfigures{!htbp}{fig:second}{Another multi-figure}}

\end{verbatim}
\end{boxedminipage}
\medskip

\afterpage{\clearpage
           \myfigures{!htbp}{fig:second}{Another multi-figure}}

\lipsum[1]

% C
\section{Use  \cs{afterpage} and \cs{clearpage} with  \cs{begin}\texttt{\{figure\}[p]}\ldots}

\afterpage{\clearpage
           \myfigures{!p}{fig:third}{Yet one figure more}}

\medskip

\begin{boxedminipage}{\textwidth}
Here we have used
\begin{verbatim}

\afterpage{\clearpage
           \myfigures{!p}{fig:third}{Yet one figure more}}

\end{verbatim}
However, this may have an undesired effect.
If there is still some figure part of the previous sequence that has not yet found a place, it will be forced out because of the \cs{clearpage} and the a new page will start, with the previous page not optimally filled, as can be seen above.
\end{boxedminipage}

\medskip 

\lipsum[1-2]

% D
\section{Use  just  \cs{begin}\texttt{\{figure\}[p]}\ldots}

\begin{boxedminipage}{\textwidth}
Here we have used
\begin{verbatim}

\myfigures{!p}{fig:fourth}{The last figure}

\end{verbatim}
This works to keep the figures on their own pages, without intervening
text, but now they are pushed far away towards the back of the document. This
is because float pages need to be reasonably full before they are
generated.
\end{boxedminipage}


\myfigures{!p}{fig:fourth}{The last figure}

\clearpage

% E
\section{Using the \texttt{subcaption} package}
\label{sec:suncaption}

The defects of the above approach are
\begin{enumerate}
\item It is clumsy to make the captions of all but the first part of the figure
\item It is hard to refer to the parts separately
\end{enumerate}
For this the \texttt{subcaption} package comes to the rescue. First it has a \cs{ContinuedFloat} command to indicate that a figure is a continuation of a previous one, and therefore will not get a new number, neither a separate entry in the list of figures if you wish.

Second, it has a \cs{subcaptionbox} command and a \texttt{subfigure} environment for the parts, where a subcaption can be given, that can also have a \cs{label} to refer to in the document. The \cs{subcaptionbox} is a specialized \cs{parbox} but its \emph{width} parameter is optional. The \texttt{subfigure} environment is a specialized \texttt{minipage}, so it has the same parameters.

These should be used in a \texttt{figure} environment, so all the placement methods of the previous section should still apply.

The \texttt{subfigure} environment has a \cs{subcaption} command for the subcaption; the \cs{subcaptionbox} has the subcaption (with its \cs{label} if desired) as its first argument. When more than one \cs{subcaptionbox} is horizontally next to each other, the subcaptions will be aligned.

In the following example we use a \cs{subcaptionbox} for the first two parts, which are together in a single \texttt{figure} environment. We use a \texttt{subfigure} environments for the other two, each one in its own \texttt{figure} environment. These use a \cs{caption}\texttt{[]\{\ldots\}}. The empty optional argument \texttt{[]} causes the caption not to appear in the list of figures.

This is the code:

\begin{verbatim}
\begin{figure}[!htbp]
  \centering
    \subcaptionbox{a subfigure in a \cs{subcaptionbox}}
      {\includegraphics[scale=0.3]{example-image-a}}
  \quad
    \subcaptionbox{another subfigure, also in a \cs{subcaptionbox}}
      {\includegraphics[scale=0.4]{example-image-b}}
  \caption{A figure with subfigures}
  \label{fig:subfigures}
\end{figure}

\begin{figure}[!htbp]\ContinuedFloat
  \begin{subfigure}{\textwidth}
    \centering
    \includegraphics[scale=0.5]{example-image-c}
    \subcaption{subfigure}
  \end{subfigure}
  \caption[]{A figure with subfigures}
\end{figure}

\begin{figure}[!htbp]\ContinuedFloat
  \begin{subfigure}{\textwidth}
    \centering 
    \includegraphics[scale=0.5]{example-image}
    \subcaption{last subfigure}
    \label{subfig:last}
  \end{subfigure}
  \caption[]{A fake caption just for demo}
\end{figure}
\end{verbatim}

\begin{figure}[!htbp]
  \centering
    \subcaptionbox{a subfigure in a \cs{subcaptionbox}}
      {\includegraphics[scale=0.3]{example-image-a}}
  \quad
    \subcaptionbox{another subfigure, also in a \cs{subcaptionbox}}
      {\includegraphics[scale=0.4]{example-image-b}}
  \caption{A figure with subfigures}
  \label{fig:subfigures}
\end{figure}

\begin{figure}[!htbp]\ContinuedFloat
  \begin{subfigure}{\textwidth}
    \centering
    \includegraphics[scale=0.5]{example-image-c}
    \subcaption{subfigure}
  \end{subfigure}
  \caption[]{A figure with subfigures}
\end{figure}

\begin{figure}[!htbp]\ContinuedFloat
  \begin{subfigure}{\textwidth}
    \centering 
    \includegraphics[scale=0.5]{example-image}
    \subcaption{last subfigure}
    \label{subfig:last}
  \end{subfigure}
  \caption[]{A fake caption just for demo}
\end{figure}

Here we show how to use the labels of both figure and subfigure.
\begin{verbatim}
The reference~\ref{subfig:last} refers to the last subfigure
of figure~\ref{fig:subfigures} on page~\pageref{subfig:last}.
\end{verbatim}

The reference~\ref{subfig:last} refers to the last subfigure
of figure~\ref{fig:subfigures} on page~\pageref{subfig:last}.

\end{document}
