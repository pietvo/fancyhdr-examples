\documentclass[twoside]{book}
\usepackage{lipsum}
\usepackage{boxedminipage}
\usepackage[a5paper]{geometry}

\usepackage{fancyhdr}

% Definitions of blobs for overview and pages.

\setlength{\unitlength}{18mm}
\newcommand{\blob}{\rule[-.2\unitlength]{2\unitlength}{.5\unitlength}}

\newcommand\rblob{\thepage
  \begin{picture}(0,0)
    \put(1,-\value{chapter}){\blob}
  \end{picture}}

\newcommand\lblob{%
  \begin{picture}(0,0)
    \put(-3,-\value{chapter}){\blob}
  \end{picture}%
  \thepage}

\pagestyle{fancy}
\fancyfoot{}

\newcounter{line}
\newcommand{\chapname}[1]{\addtocounter{line}{1}%
  \put(1,-\value{line}){\blob}
  % Adjust these numbers so the text gets the proper indentation
  \put(-5.5,-\value{line}){\Large \arabic{line}}
  \put(-5,-\value{line}){\Large #1}}

\newcommand{\overview}{%
  \begin{picture}(0,0)
    \chapname{Introduction}
    \chapname{Another chapter}
    \chapname{Third case}
    \chapname{Fourth case}
    \chapname{Fifth case}
    \chapname{Sixth case}
    \chapname{Seventh case}
  \end{picture}}

\begin{document}

% A normal book would have a title page, and other front matter first.

% Then the overview page
% The page doesn't have 'contents'; all the visual contents is generated
% by the \overview command in the header

\fancyhead[L]{Overview}
\fancyhead[R]{\overview}
\mbox{}\newpage % This produces the overview page

% Front matter -- doesn't have blobs.

\fancyhead[RE]{\rightmark}
\fancyhead[RO,LE]{}
\fancyhead[LO]{\leftmark}

\pagenumbering{roman}
\thispagestyle{plain}
\tableofcontents

\bigskip

\noindent
\begin{boxedminipage}{\textwidth}
This is example 31 in the \texttt{fancyhdr} documentation.

It shows how to generate a `thumb index'.

\begin{verbatim}
\setlength{\unitlength}{18mm}
\newcommand{\blob}{%
  \rule[-.2\unitlength]{2\unitlength}{.5\unitlength}}

\newcommand\rblob{\thepage
  \begin{picture}(0,0)
    \put(1,-\value{chapter}){\blob}
  \end{picture}}

\newcommand\lblob{%
  \begin{picture}(0,0)
    \put(-3,-\value{chapter}){\blob}
  \end{picture}%
  \thepage}

\pagestyle{fancy}
\fancyfoot{}

\newcounter{line}
\newcommand{\chapname}[1]{\addtocounter{line}{1}%
  \put(1,-\value{line}){\blob}
  % Adjust these numbers for the proper indentation
  \put(-5.5,-\value{line}){\Large \arabic{line}}
  \put(-5,-\value{line}){\Large #1}}

\newcommand{\overview}{%
  \begin{picture}(0,0)
    \chapname{Introduction}
    \chapname{Another chapter}
    \chapname{Third case}
    \chapname{Fourth case}
    \chapname{Fifth case}
    \chapname{Sixth case}
    \chapname{Seventh case}
  \end{picture}}
\end{verbatim}
\end{boxedminipage}

\noindent
\begin{boxedminipage}{\textwidth}
The overview page:\\
The page doesn't have 'contents' --
all the visual contents is generated
by the \verb|\overview| command in the header

\begin{verbatim}
\fancyhead[L]{Overview}
\fancyhead[R]{\overview}
\mbox{}\newpage % This produces the overview page

% Front matter -- doesn't have blobs.

\fancyhead[RE]{\rightmark}
\fancyhead[RO,LE]{}
\fancyhead[LO]{\leftmark}

\pagenumbering{roman}
\thispagestyle{plain}
\tableofcontents
 . . .
\newpage

\pagenumbering{arabic}

\fancyhead[RO]{\rblob}
\fancyhead[LE]{\lblob}

% Page style 'plain' does not have the usual header,
% but it does have the blobs.

\fancypagestyle{plain}{%
  \fancyhead[RE,LO]{}
  \renewcommand{\headrule}{}%
}
\end{verbatim}
\end{boxedminipage}

\newpage

% Here the document begins

\pagenumbering{arabic}

% Now activate the blobs

\fancyhead[RO]{\rblob}
\fancyhead[LE]{\lblob}

% Page style 'plain' does not have the usual header,
% but it does have the blobs.

\fancypagestyle{plain}{%
  \fancyhead[RE,LO]{}
  \renewcommand{\headrule}{}%
}

% Here the document begins

\chapter{Introduction}

\lipsum[1-4]

\section{The Problem}
\label{sec:problem}

  \lipsum

\section{Evaluation}

\lipsum[3-6]

\chapter{Another chapter}

\label{cha:another-chapter}

\lipsum

\section{Another section}

\lipsum[3-4]

\chapter{Third case}

\lipsum[1-2]

\section{Third section}

\lipsum

\chapter{Fourth case}

\lipsum[1-2]

\section{Fourth section}

\lipsum

\chapter{Fifth case}

\lipsum[1-2]

\section{fifth section}

\lipsum

\chapter{Sixth case}

\lipsum[1-2]

\section{Sixth section}

\lipsum

\chapter{Seventh case}

\lipsum[1-2]

\section{Seventh section}

\lipsum

\end{document}
