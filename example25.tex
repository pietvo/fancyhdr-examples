\documentclass[openany]{book}
\usepackage{lipsum}
\usepackage{boxedminipage}
\usepackage{ifthen}
\usepackage{fancyhdr}
\pagestyle{fancy}
\fancyhead[LE,RO]{\textsl{\rightmark}}
\fancyhead[LO,RE]{\textsl{\leftmark}}
\fancyfoot[C]{\thepage}
\renewcommand{\chaptermark}[1]{\markboth{#1}{}}
\renewcommand{\sectionmark}[1]{\markright{#1}}

\fancypagestyle{myfancy}{
  \fancyhead[LE,RO]{MYFANCY SECTION}
% The rest is inherited from fancy
}

\fancypagestyle{switch}{
  \fancyhead[LE,RO]{%
    \ifthenelse{\value{page}=\pageref{otherpagestyle}}
      {MYFANCY SECTION}
      {\textsl{\rightmark}}}
% The rest is inherited from pagestyle fancy
}

\newcommand{\strutheader}{%
  \texttt{\textbackslash strut}:
  \rule[-0.3\normalbaselineskip]{10pt}{0.3\normalbaselineskip}%
  \rule{10pt}{0.7\normalbaselineskip}
  \texttt{\textbackslash headruleskip}$\searrow$
  \raisebox{-0.3\normalbaselineskip}[0pt][0pt]%
    {\ifdim \headruleskip>0pt
      \rule[-\headruleskip]{10pt}{\headruleskip}
      \else
      \rule{10pt}{-\headruleskip}
     \fi}
}
\newcommand{\strutfooter}{%
  \raisebox{0pt}[0pt][0pt]{%
    \texttt{\textbackslash strut}:
    \rule[-0.3\normalbaselineskip]{10pt}{0.3\normalbaselineskip}%
    \rule{10pt}{0.7\normalbaselineskip}
    \texttt{\textbackslash footruleskip} $\nearrow$
    \rule[0.7\normalbaselineskip]{10pt}{\footruleskip}}%
}

\fancypagestyle{showstruts}{%
  \fancyhead[L]{%
    \ifthenelse{\value{page}=\pageref{showstruts}}%
      {\strutheader}%
      {\rightmark}%
  }
  \fancyfoot[L]{%
    \ifthenelse{\value{page}=\pageref{showstruts}}%
      {\strutfooter}%
      {}%
  }
  \fancyheadinit{%
    \ifthenelse{\value{page}=\pageref{showstruts}}%
      {\renewcommand{\headruleskip}{4pt}}%
      {\renewcommand{\headruleskip}{0pt}}%
  }
  \fancyfootinit{%
    \ifthenelse{\value{page}=\pageref{showstruts}}%
      {\renewcommand{\footrulewidth}{0.4pt}}%
      {\renewcommand{\footrulewidth}{0pt}}%
  }
}

\newcommand{\cs}[1]{\texttt{\char`\\#1}}

\begin{document}
\tableofcontents
\newpage
\thispagestyle{plain}
\noindent
\begin{boxedminipage}{\textwidth}
This is example 25 in the fancyhdr documentation.

In chapter~\ref{ch:just-thispagestyle} it demonstrates \cs{thispagestyle}, both a good and a bad example.

In chapter~\ref{ch:pageref} it gives a better solution to change the page style in the middle of a text. This is example 25~(a) in the \textsf{fancyhdr} documentation.

Finally, chapter~\ref{ch:showstruts} gives the code for the ``\emph{showstruts}'' example 25~(b) in the documentation.

The normal page style in this document is:
\begin{verbatim}
\pagestyle{fancy}
\fancyhead[LE,RO]{\textsl{\rightmark}}
\fancyhead[LO,RE]{\textsl{\leftmark}}
\fancyfoot[C]{\thepage}
\renewcommand{\chaptermark}[1]{\markboth{#1}{}}
\renewcommand{\sectionmark}[1]{\markright{#1}}
\end{verbatim}
We use the following additional page style \texttt{myfancy}. It just puts ``MYFANCY SECTION'' instead of the section title:
\begin{verbatim}
\fancypagestyle{myfancy}{
  \fancyhead[LE,RO]{MYFANCY SECTION}
% The rest is inherited from fancy
}
\end{verbatim}

\end{boxedminipage}

\chapter{Just \cs{thispagestyle}}
\label{ch:just-thispagestyle}

\thispagestyle{myfancy}

\begin{boxedminipage}{\textwidth}
Here we start with a \verb|\thispagestyle{myfancy}|. The header on this page comes out as\\
on the left: \textsl{Just \cs{thispagestyle}} (=the normal header)\\
on the right: MYFANCY SECTION

The next page will have a more traditional header with chapter and section titles.

The  \verb|\thispagestyle{myfancy}| works here, because it is given right at the beginning of the chapter where we are sure that a page break will not occur.
\end{boxedminipage}

\bigskip

\lipsum[1-3]
\lipsum[3-5]

\newpage

\section{A wrong \cs{thispagestyle} attempt}

\begin{boxedminipage}{\textwidth}
In this section we will put a \verb|\thispagestyle{myfancy}| in the middle of the text. But it appears that this text is close to the end of the page. So in this case \TeX{} is still on the current page when it encounters the \verb|\thispagestyle{myfancy}| command. That means \textbf{the MYFANCY SECTION header comes at the current page}. However, during the page breaking \TeX{} decides that the text will not fit on the page and moves it to the next page. \textbf{Now the different header is no longer at the page where that text is.}
\end{boxedminipage}

\bigskip

\lipsum[6-8]

\bigskip
Suspendisse vel felis. Ut lorem lorem, interdum eu, tincidunt sit amet, laoreet vitae, arcu. Aenean faucibus pede eu ante. Praesent enim elit, rutrum at, molestie non, nonummy vel, nisl. Ut lectus eros, malesuada sit amet, fer- mentum eu, sodales cursus, magna. Donec eu purus. Quisque vehicula, urna sed ultricies auctor, pede lorem egestas dui, et convallis elit erat sed nulla. Donec luctus. Curabitur et nunc. Aliquam dolor odio, commodo pretium, ultricies non, pharetra in, velit. Integer arcu est, nonummy in, fermentum faucibus, egestas vel, odio.

\bigskip

\noindent
\begin{boxedminipage}{\textwidth}
\textbf{This page was supposed have a MYFANCY SECTION header. 
\thispagestyle{myfancy}
It is done with the} \verb|\thispagestyle{myfancy}| \textbf{command right at this place.
However, it appears that the MYFANCY SECTION header comes on the previous page.}
\end{boxedminipage}

\bigskip

\lipsum[8-9]

\newpage

\chapter{A better solution}
\label{ch:pageref}
\pagestyle{switch}

\begin{boxedminipage}{\textwidth}
This is example 25~(a).

In this chapter we use a better solution to the \verb|\thispagestyle{myfancy}| in the middle of the text.

For selecting a different page style for the page where a specific text appears we put a label \texttt{otherpagestyle} in the text and then use a definition in the header that chooses

 If the page number is equal to the page number of the label we use the same as \verb|\pagestyle{myfancy}|, otherwise the default values of \verb|\pagestyle{fancy}|.

\begin{verbatim}
\fancypagestyle{switch}{
  \fancyhead[LE,RO]{%
    \ifthenelse{\value{page}=\pageref{otherpagestyle}}
      {MYFANCY SECTION}
      {\textsl{\rightmark}}}
% The rest is inherited from fancy
}
\end{verbatim}
\end{boxedminipage}

\bigskip

\lipsum[6-8]

\bigskip
Suspendisse vel felis. Ut lorem lorem, interdum eu, tincidunt sit amet, laoreet vitae, arcu. Aenean faucibus pede eu ante. Praesent enim elit, rutrum at, molestie non, nonummy vel, nisl. Ut lectus eros, malesuada sit amet, fer- mentum eu, sodales cursus, magna. Donec eu purus. Quisque vehicula, urna sed ultricies auctor, pede lorem egestas dui, et convallis elit erat sed nulla. Donec luctus. Curabitur et nunc. 

\section{Here it comes}

\noindent
\begin{boxedminipage}{\textwidth}
\begin{minipage}{\linewidth}
  \textbf{This page should have a different header than the surrounding
    pages. The header should contain \textnormal{``MYFANCY SECTION''}.
    \label{otherpagestyle}
    It is done with the} \verb|\pagestyle{switch}| \textbf{command, that
    has tests in the header field definitions. This chooses the actual
    header depending on the page number.}
\end{minipage}
\end{boxedminipage}
\bigskip

\lipsum[8-10]

\medskip

\noindent
\begin{boxedminipage}{\textwidth}
\textbf{This page should get the normal header again because the page number is different from the page number in the label.}
\end{boxedminipage}

\chapter{Showstruts}
\label{ch:showstruts}
\pagestyle{showstruts}

\begin{boxedminipage}{\textwidth}
This is example 25~(b) from the \textsf{fancyhdr} documentation. It uses the same principle as the previous chapter to get a different header on one particular page.
Only the page style is now \texttt{showstruts} and also the label is called \texttt{showstruts}.
We start the \verb|\pagestyle{showstruts}| right here at the beginning of the chapter.

Here is the code:

\begin{verbatim}
\newcommand{\strutheader}{%
  \texttt{\textbackslash strut}:
  \rule[-0.3\normalbaselineskip]{10pt}{0.3\normalbaselineskip}%
  \rule{10pt}{0.7\normalbaselineskip}
  \texttt{\textbackslash headruleskip}$\searrow$
  \raisebox{-0.3\normalbaselineskip}[0pt][0pt]%
    {\ifdim \headruleskip>0pt
      \rule[-\headruleskip]{10pt}{\headruleskip}
      \else
      \rule{10pt}{-\headruleskip}
     \fi}
}
\newcommand{\strutfooter}{%
  \raisebox{0pt}[0pt][0pt]{%
    \texttt{\textbackslash strut}:
    \rule[-0.3\normalbaselineskip]{10pt}{0.3\normalbaselineskip}%
    \rule{10pt}{0.7\normalbaselineskip}
    \texttt{\textbackslash footruleskip} $\nearrow$
    \rule[0.7\normalbaselineskip]{10pt}{\footruleskip}}%
}
\end{verbatim}
\end{boxedminipage}

\noindent
\begin{boxedminipage}{\textwidth}
\begin{verbatim}
\fancypagestyle{showstruts}{%
  \fancyhead[L]{%
    \ifthenelse{\value{page}=\pageref{showstruts}}%
      {\strutheader}%
      {\rightmark}%
  }
  \fancyfoot[L]{%
    \ifthenelse{\value{page}=\pageref{showstruts}}%
      {\strutfooter}%
      {}%
  }
  \fancyheadinit{%
    \ifthenelse{\value{page}=\pageref{showstruts}}%
      {\renewcommand{\headruleskip}{4pt}}%
      {\renewcommand{\headruleskip}{0pt}}%
  }
  \fancyfootinit{%
    \ifthenelse{\value{page}=\pageref{showstruts}}%
      {\renewcommand{\footrulewidth}{0.4pt}}%
      {\renewcommand{\footrulewidth}{0pt}}%
  }
}
\end{verbatim}
\end{boxedminipage}

\section{Here we go}

\lipsum[1-3]

\medskip
\noindent
\begin{boxedminipage}{\textwidth}
\textbf{Here we put the label \texttt{showstruts}, so the struts in the header and footer should appear on this page.}
\label{showstruts}
\end{boxedminipage}
\medskip

\lipsum[3-5]

\end{document}
